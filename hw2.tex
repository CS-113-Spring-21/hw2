\documentclass[addpoints]{exam}

\usepackage{amsmath}
\usepackage{amssymb}
\usepackage{geometry}
\usepackage{tabularx}

% Header and footer.
\pagestyle{headandfoot}
\runningheadrule
\runningfootrule
\runningheader{CS 113 Discrete Mathematics}{HW 2: Logic}{Spring 2020}
\runningfooter{}{Page \thepage\ of \numpages}{}
\firstpageheader{}{}{}

% \qformat{{\large\bf \thequestion. \thequestiontitle}\hfill[\totalpoints\ points]}
\boxedpoints
\printanswers

\newcommand\ol\overline

\title{Homework 2: Logic\\ CS 113 Discrete Mathematics\\ Habib University -- Spring 2020}
\author{Don't Grade Me}  % replace with your ID, e.g. oy02945
\date{}

\begin{document}
\maketitle

\begin{questions}
  
  
\question De Morgan's Laws from set theory have their counterparts in logic. Prove each of these using truth tables.
  \begin{parts}
  \part[5] Prove that $\neg(p \lor q) \equiv \neg p \land \neg q $.
    \begin{solution}
      % Complete  the solution.
      \[
        \begin{array}{c|c|*{6}{|c}}
          p & q & p \lor q & \neg(p \lor q) & \neg p & \neg q & \neg p \land \neg q & (\neg(p \lor q)) \iff (\neg p \land \neg q)\\
          \hline
          F & F & F & T & T & T & T & T \\
          F & T & T & F & T & F & F & T \\
          T & F & T & F & F & T & F & T \\
          T & T & T & F & F & F & F & T 
        \end{array}
      \]
      We see that $(\neg(p \lor q)) \iff (\neg p \land \neg q)$ is a tautology.\\
      $\implies (\neg(p \lor q)) \equiv (\neg p \land \neg q)$.\\
      Hence proved.
    \end{solution}
  \part[5] Prove that $\neg(p \land q) \equiv \neg p \lor \neg q $.
    \begin{solution}
      % Write your solution here
      \[
        \begin{array}{c|c|*{6}{|c}}
          p & q & p \land q & \neg(p \land q) & \neg p & \neg q & \neg p \lor \neg q & (\neg(p \land q)) \iff (\neg p \lor \neg q)\\
          \hline
          F & F & F & T & T & T & T & T \\
          F & T & F & T & T & F & T & T \\
          T & F & F & T & F & T & T & T \\
          T & T & T & F & F & F & F & T 
        \end{array}
      \]
      We see that $(\neg(p \land q)) \iff (\neg p \lor \neg q)$ is a tautology.\\
      $\implies (\neg(p \land q)) \equiv (\neg p \lor \neg q)$.\\
      Hence proved.
    \end{solution}
  \end{parts}

\question We want to write the statement, ``A person is popular only if they are cool or funny'', in propositional logic.
  \begin{parts}
  \part[5] Identify three simple propositions, $p, q, \text{ and } r$, needed for the representation and write out the corresponding expression that uses them to represent the given sentence.
    \begin{solution} % Complete the solution.
      The simple propositions are as follows.\\
      \begin{tabularx}{\textwidth}{l@{ : }X}
        $p$ & The person is popular.\\
        $q$ & The person is cool.\\
        $r$ & The person is funny.
      \end{tabularx}
      
      The expression is : $p \implies (q\lor r)$.
    \end{solution}
  \part[5] For your expression identified above, write the converse, contrapositive, and inverse in propositional logic as well as complete English sentences.
    \begin{solution} % Complete the solution.
      The expression is : .\\
      \begin{tabularx}{\textwidth}{l|l|X}
        & Logical Notation & English sentence \\\hline\hline
        Converse &  $(q\lor r) \implies p$ & A person is cool or funny only if they are popular\\\hline
        Contrapositive & $(\lnot q\land \lnot r) \implies \lnot p$ & A person is not cool and not funny only if they are not popular\\\hline
        Inverse & $\lnot p \implies (\lnot q\land \lnot r)$ &  A person is not popular only if they are neither funny nor cool
      \end{tabularx}
      Note that there are other correct ways to express the above.
    \end{solution}
  \end{parts}

\question[5] A small company makes widgets in a variety of constituent materials (aluminum, copper, iron), colors (red, green, blue, grey), and finishes (matte, textured, coated). Although there are many combinations of widget features, the company markets only a subset of the possible combinations. The following sentences are constraints that characterize the possibilities. 
  \begin{enumerate}
  \item aluminum $\lor$ copper $\lor$ iron
  \item aluminum $\implies$ grey
  \item copper $\land$ $\neg$ coated $\implies$ red
  \item coated $\land$ $\neg$ copper $\implies$ green
  \item green $\lor$ blue $\iff \neg$ textured $\land$ $\neg$ iron
  \end{enumerate}
  Suppose that a customer places an order for a copper widget that is both green and blue with a matte coating. Your job is to determine which constraints are satisfied and which are violated and provide an explanation.
  \begin{solution} % Complete the solution.
    The order can be represented as: copper $\land$ green  $\land$ blue  $\land$ matte.\\
    That is, the truth value of each of the above is True. The truth value of all other propositions is False.
    
    In the table below, a value of 1 for a constraint indicates that the constraint is satisfied and a value of 0 indicates that it is violated.

    \begin{tabularx}{\textwidth}{l|l|X}
      Constraint & Satisfied & Explanation \\\hline\hline
      aluminum $\lor$ copper $\lor$ iron & 1 & The constraint is satisfied because copper is T.\\\hline
      aluminum $\implies$ grey & 1 & The implication is satisfied because both the antecedent and consequent are F. \\\hline
      copper $\land$ $\neg$ coated $\implies$ red & 0 & The implication is not satisfied because the antecedent is T but the consequent is F. \\\hline
      coated $\land$ $\neg$ copper $\implies$ green & 1 & The implication is satisfied because both the antecedent is F and the consequent is T. \\\hline
      green $\lor$ blue $\iff \neg$ textured $\land$ $\neg$ iron & 1 & The biconditional is satisfied because both sides have a truth value of T.
    \end{tabularx}
  \end{solution}
  
\question Consider the following statements.
  \begin{itemize}
  \item $ (p \lor q) \implies \neg r$
  \item $(\neg p \land \neg q) \land \neg r$
  \end{itemize}
  \begin{parts}
  \part[5] Prove their logical equivalence using a truth table.
    \begin{solution} For ease of notation, let
      \[
        \begin{array}{l@{\text{ : }} l}
          A & (p\lor q)\implies \lnot r \\
          B & (\lnot p \land \lnot q) \land \lnot r
        \end{array}
      \]
      So we have to prove that $A \equiv B$.
      \[
        \begin{array}{*{3}{c|}*{8}{|c}}
          p & q & r & \lnot p & \lnot q & \lnot r & p \lor q & \lnot p \land \lnot q & A & B & A \iff B\\
          \hline
          F & F & F & T & T & T & F & T & T & T & T \\
          F & F & T & T & T & F & F & T & T & F & F \\
          F & T & F & T & F & T & T & F & T & F & F \\
          F & T & T & T & F & F & T & F & F & F & T \\
          T & F & F & F & T & T & T & F & T & F & F \\
          T & F & T & F & T & F & T & F & F & F & T \\
          T & T & F & F & F & T & T & F & T & F & F \\
          T & T & T & F & F & F & T & F & F & F & T 
        \end{array}
      \]
      We see that $A\iff B$ is not a tautology.\\
      $\implies A \not\equiv B$.
    \end{solution}
  \part[5] If they are not equivalent, propose a statement obtained after minimal modification to the second one such that it is equivalent to the first. Prove the equivalence using identities.
    \begin{solution}
      $ B\colon (\lnot p \land \lnot q) \land \lnot r$ can be modified to $C\colon (\lnot p \land \lnot q) \lor \lnot r$. Then $A\equiv C$.\\
      Proof:\\
      \begin{align*}
        A\colon  (p\lor q)\implies \lnot r & \equiv \lnot (p \lor q) \lor \lnot r & \text{using } p\implies q \equiv \lnot p \lor q\\
        \; & \equiv (\lnot p \land \lnot q) \lor \lnot r & \text{using De Morgan's law}\\
        \; & \equiv C & \text{using definition of } C\\
      \end{align*}
      Hence proved.
    \end{solution}
  \end{parts}

  
\question A proposition that is always \textit{True} regardless of the truth values of the simpler propositions involved is called a \textit{tautology}. In other words, a tautology is a proposition that is logically equivalent to \textit{True}.
  \begin{parts}
  \part[5] Prove that $((p \implies q) \land (q \implies r)) \implies (p \implies r)$ is a tautology.
    \begin{solution}
      For ease of notation, let us use the following alternate representation.
      \[
        \lnot p\colon \ol{p},\quad p \land q\colon p.q, \quad p \lor q\colon p+q
      \]
      Then,
      \begin{align*}
        & ((p \implies q) \land (q \implies r)) \implies (p \implies r)\\
        & \equiv ((\ol{p} + q).  (\ol{q} + r)) \implies (\ol{p} +  r) & (p\rightarrow q) \equiv (\ol{p}+q)\\
        & \equiv \ol{(\ol{p}.\ol{q} + \ol{p}.r + q.\ol{q} + q.r)} + \ol{p} +  r & \text{distribution, implication}\\
        & \equiv \ol{(\ol{p}.\ol{q} + \ol{p}.r + F + q.r)} + \ol{p} +  r & p.\ol{p} \equiv F\\
        & \equiv \ol{(\ol{p}.\ol{q} + \ol{p}.r + q.r)} + \ol{p} +  r & p + F \equiv p\\
        & \equiv \ol{\ol{p}.\ol{q}}.\ol{ \ol{p}.r}.\ol{q.r} + \ol{p} +  r & \text{De Morgan's law}\\
        & \equiv (p+q).(p+\ol{r}).(\ol{q} + \ol{r}) + \ol{p} +  r & \text{De Morgan's law}\\
        & \equiv (p.p + p.\ol{r} +p.q + q.\ol{r}).(\ol{q} + \ol{r}) + \ol{p} +  r & \text{distribution}\\
        & \equiv (p + p.\ol{r} +p.q + q.\ol{r}).(\ol{q} + \ol{r}) + \ol{p} +  r & p.p\equiv p\\
        & \equiv (p +p.q + q.\ol{r}).(\ol{q} + \ol{r}) + \ol{p} +  r & \text{absorption}\\
        & \equiv (p + q.\ol{r}).(\ol{q} + \ol{r}) + \ol{p} +  r & \text{absorption}\\
        & \equiv (p.\ol{q} + p.\ol{r} + q.\ol{q}.\ol{r}  + q.\ol{r}.\ol{r}) + \ol{p} +  r & \text{distribution}\\
        & \equiv (p.\ol{q} + p.\ol{r} + F.\ol{r}  + q.\ol{r}) + \ol{p} +  r & p.\ol{p} \equiv F\\
        & \equiv (p.\ol{q} + p.\ol{r}  + q.\ol{r}) + \ol{p} +  r & p.F\equiv F, p+F\equiv p\\
        & \equiv (r + p.\ol{r})  + (\ol{p} + p.\ol{q}) +q.\ol{r} & \text{rearrange}\\
        & \equiv r + p  + \ol{p} + \ol{q} + q.\ol{r} & p+\ol{p}.q\equiv p+q\\
        & \equiv r + T + \ol{q} + q.\ol{r} & p + \ol{p}\equiv T\\
        & \equiv T & p+T\equiv T
      \end{align*}
      Hence proved.
    \end{solution}
    \begin{solution}
      \begin{align*}
        & ((p \implies q) \land (q \implies r)) \implies (p \implies r)\\
        & \equiv ((\lnot p \lor  q) \land (\lnot q \lor  r)) \implies (\lnot p \lor   r) & (p\rightarrow q) \equiv (\lnot p\lor q)\\
        & \equiv \lnot ((\lnot p\land \lnot q) \lor  (\lnot p\land r) \lor  (q\land \lnot q) \lor  (q\land r)) \lor  \lnot p \lor   r & \text{distribution, implication}\\
        & \equiv \lnot ((\lnot p\land \lnot q) \lor  (\lnot p\land r) \lor  F \lor  (q\land r)) \lor  \lnot p \lor r & p\land \lnot p \equiv F\\
        & \equiv \lnot((\lnot p\land \lnot q) \lor  (\lnot p\land r) \lor  (q\land r)) \lor  \lnot p \lor   r & p \lor  F \equiv p\\
        & \equiv (\lnot(\lnot p\land \lnot q)\land \lnot( \lnot p\land r)\land \lnot(q\land r)) \lor  \lnot p \lor   r & \text{De Morgan's law}\\
        & \equiv ((p\lor q)\land (p\lor \lnot r)\land (\lnot q \lor  \lnot r)) \lor  \lnot p \lor r & \text{De Morgan's law}\\
        & \equiv (((p\land p) \lor  (p\land \lnot r) \lor (p\land q) \lor  (q\land \lnot r))\land (\lnot q \lor  \lnot r)) \lor  \lnot p \lor   r & \text{distribution}\\
        & \equiv ((p \lor  (p\land \lnot r) \lor (p\land q) \lor  (q\land \lnot r))\land (\lnot q \lor  \lnot r)) \lor  \lnot p \lor   r & p\land p\equiv p\\
        & \equiv ((p \lor (p\land q) \lor  (q\land\lnot r))\land (\lnot q \lor  \lnot r)) \lor  \lnot p \lor   r & \text{absorption}\\
        & \equiv ((p \lor  (q\land \lnot r))\land (\lnot q \lor  \lnot r)) \lor  \lnot p \lor   r & \text{absorption}\\
        & \equiv ((p\land \lnot q) \lor  (p\land \lnot r) \lor  (q\land \lnot q\land \lnot r)  \lor  (q\land \lnot r\land \lnot r)) \lor  \lnot p \lor   r & \text{distribution}\\
        & \equiv (p\land \lnot q) \lor  (p\land \lnot r) \lor  (F\land \lnot r)  \lor  (q\land \lnot r) \lor  \lnot p \lor   r & p\land \lnot p \equiv F\\
        & \equiv (p\land \lnot q) \lor  (p\land \lnot r) \lor  (q\land \lnot r) \lor  \lnot p \lor   r & p\land F\equiv F, p\lor F\equiv p\\
        & \equiv (r \lor  (p\land \lnot r))  \lor  (\lnot p \lor  (p\land \lnot q)) \lor q\land \lnot r & \text{rearrange}\\
        & \equiv r \lor  p  \lor  \lnot p \lor  \lnot q \lor  q\land \lnot r & p\lor \lnot p\land q\equiv p\lor q\\
        & \equiv r \lor  T \lor  \lnot q \lor  q\land\lnot r & p \lor  \lnot p\equiv T\\
        & \equiv T & p\lor T\equiv T
      \end{align*}
      Hence proved.
    \end{solution}
    \begin{solution} Assume the following notation.
      \[
        \begin{array}{l@{\text{ : }}l}
          A & p \implies q\\
          B & q \implies r\\
          C & p \implies r
        \end{array}
      \]
      Then we have to show that $(A \land B) \implies C$.
      \[
        \begin{array}{*{3}{c|}*{5}{|c}}
          p & q & r & A & B & A \land B & C & (A \land B) \implies C\\
          \hline
          F & F & F & T & T & T & T & T \\
          F & F & T & T & T & T & T & T \\
          F & T & F & T & F & F & T & T \\
          F & T & T & T & T & T & T & T \\
          T & F & F & F & T & F & F & T \\
          T & F & T & F & T & F & T & T \\
          T & T & F & T & F & F & F & T \\
          T & T & T & T & T & T & T & T 
        \end{array}
      \]
      $\therefore (A \land B) \implies C$ is a tautology.\\
      Hence proved.
    \end{solution}
    
  \part[5] Given propositions, $p$ and $q$, $q$ is said to be \textit{inferred} from $p$ if $p \implies q$ is a tautology. Write a statement in English that correctly applies this new terminology to the expression in the previous part. Also provide an example in English by assigning suitable propositions to $p,q, \text{ and } r$.
    \begin{solution} % Complete the solution.
      A statement of the expression from the previous part is:\\
      \centerline{If $q$ is inferred from $p$ and $r$ is inferred from $q$, then $r$ is inferred from $p$.} 
      Using the following propositions for  $p,q, \text{ and } r$,\\
      \begin{tabularx}{\textwidth}{l@{ : }X}
        $p$ & It is raining.\\
        $q$ & Streets are wet.\\
        $r$ & Cars can skid.\\
      \end{tabularx}
      the statement becomes:
      \quote{If it is raining, the streets are wet and if the streets are wet then cars can skid. Therefore, if it is raining, cars can skid.}
    \end{solution}
  \end{parts}


\question[5] You are given four cards each of which has a number on one side and a letter on another. You place them on a table in front of you and the four cards read: $A\ 5\ 2\ J$. Which cards would you turn over in order to test the following rule? 
  \begin{center}
    Cards with $5$ on one side have $J$ on the other side.
  \end{center}
  Explain your choice.
  \begin{solution} The rule can be written as $5\implies J$, where the antecedent indicates the sign on one side and the consequent indicates the sign on the other side of the card.
    
    In the table below, a value of 1 for a card indicates that I will turn it and a value of 0 indicates that I will not turn it.
    
    \begin{tabularx}{\textwidth}{c|c|X}
      Card & Turned & Explanation \\\hline\hline
      $A$ & 1 & The consequent of the implication is False. Whenever this is case, the implication's truth value depends on the value of the antecedent. If the antecedent is True, i.e. there is a 5 on the other side, the implication will be False. And if the antecedent is False, i.e. there is not a 5 on the other side, the implication will be True. Therefore this card needs to be turned to test the validity of the rule.\\\hline
      $5$ & 1 & The antecedent of the implication is True. The truth value of the implication in this case is the same as that of the consequent. So the letter on the other side will impact the validity of the rule. If the letter on the other side is $J$, the implication will be True, otherwise the implication will be False. Therefore this card needs to be turned to test the validity of the rule.\\\hline
      $2$ & 0 & The antecedent of the implication is False. Whenever this is case, the implication is True regardless of the truth value of the consequent. So, the letter on the other side of the card will make no difference on the validity of the rule.\\\hline
      $J$ & 0 & The consequent of the implication is True. Whenever this is case, the implication is True regardless of the truth value of the antecedent. So, the number on the other side of the card will make no difference on the validity of the rule.
    \end{tabularx}
  \end{solution}
  
\question An argument is said to be \textit{valid} if its \textit{conclusion} can be inferred from its \textit{premises}. An argument that is not valid is called an \textit{invalid} argument, or a \textit{fallacy}. For each of the arguments below, identify the simple propositions involved, write the premises and conclusion(s) in logical notation using the identified simple propositions, and decide whether it is valid. Justify your decision.
  \begin{parts}
  \part[5] If I am wealthy, then I am happy. I am happy, therefore, I am wealthy.
    \begin{solution}
      The simple propositions are as follows.\\
      \begin{tabularx}{\textwidth}{l@{ : }X}
        $p$ & I am wealthy.\\
        $q$ & I am happy.
      \end{tabularx}

      The argument is
      \[
        \begin{array}{l}
          p\implies q\\
          q\\\hline
          p\\
        \end{array}
      \]
      Given the implication, the consequent $q$ can be inferred from the antecedent $p$, and as per the contrapositive, which is logically equivalent, $\lnot p$ can be inferred from $\lnot q$. There is no information about any inference from $q$ and the conclusion $p$ \textit{cannot} be inferred from the premises. Therefore, this argument is invalid. This form of fallacy is common and is called, ``affirming the consequent''.

      Another proof approach could be to test whether $((p\implies q)\land q) \implies p$ is a tautology.
    \end{solution}
  \part[5]
    If Ahmed drives his car, he is at least 18 years old. Ahmed does not drive a car. Therefore, Ahmed is not yet 18 years old. 
    \begin{solution}
      The simple propositions are as follows.\\
      \begin{tabularx}{\textwidth}{l@{ : }X}
        $p$ & Ahmed drives his car.\\
        $q$ & Ahmed is at least 18 years old.
      \end{tabularx}

      The argument is
      \[
        \begin{array}{l}
          p\implies q\\
          \lnot p\\\hline
          \lnot q\\
        \end{array}
      \]
      Given the implication, the consequent $q$ can be inferred from the antecedent $p$, and as per the contrapositive, which is logically equivalent, $\lnot p$ can be inferred from $\lnot q$. There is no information about any inference from $\lnot p$ and the conclusion $\lnot q$ \textit{cannot} be inferred from the premises. Therefore, this argument is invalid. This form of fallacy is common and is called, ``denying the antecedent''.

      Another proof approach could be to test whether $((p\implies q)\land \lnot p) \implies \lnot q$ is a tautology.
    \end{solution}
  \part[5] If I study, then I will not fail CS 113. If I do not play cards too often, then I will study. I failed CS 113. Therefore, I played cards too often.
    \begin{solution}
      The simple propositions are as follows.\\
      \begin{tabularx}{\textwidth}{l@{ : }X}
        $p$ & I study.\\
        $q$ & I will fail CS 113.\\
        $r$ & I play cards too often.
      \end{tabularx}

      The argument is
      \[
        \begin{array}{l@{\hspace{50pt}}r}
          p\implies \lnot q & (1)\\
          \lnot r \implies p & (2)\\
          q & (3)\\\hline
          r & 
        \end{array}
      \]
      
      Proof of validity:
      \[
        \begin{array}{l|l@{\hspace{50pt}}r}
          \text{Inference} & \text{Premises} & \\\hline
          \lnot p & (1) \land (3) & (4)\\
          r & (2) \land (4) & \\
        \end{array}
      \]
      Hence proved.
    \end{solution}
  \end{parts}

\question[5] One of your TA's has hidden a manual titled, ``Sacred Secrets: How to Earn an A+ and Keep your Mind'', somewhere on campus. As they could themselves not benefit from this manual, the directions they have left for you to find the manual are as follows.
  \begin{enumerate}
  \item There is a hint at Learn Courtyard or at the Gym.
  \item If your TA is sitting in Ehsas or they are absent, then there is a hint at Learn Courtyard.
  \item If your TA is not sitting in Ehsaas, then there is a hint at the Gym.
  \item If there are people in Learn Courtyard, then there is no hint at Learn Courtyard.
  \item If there is a hint at Learn Courtyard, then the manual is at Zen Garden.
  \item If there is hint at the Gym, then the manual is at Earth Courtyard.
  \item If your TA is absent, then the manual is at Fire Courtyard.
  \end{enumerate}
  You notice that there are people in Learn Courtyard. Where is the manual?

  Identify the relevant simple propositions to model the above in propositional logic. Represent the above situation using propositional logic and describe the steps needed to infer the location of the manual.
  \begin{solution}
    The simple propositions are as follows.\\
    \begin{tabularx}{\textwidth}{l@{ : }X}
      $p$ & There is a hint at Learn Courtyard.\\
      $q$ & There is a hint at the Gym.\\
      $r$ & Your TA is sitting in Ehsas.\\
      $s$ & Your TA is absent.\\
      $t$ & There are people in Learn Courtyard.\\
      $u$ & The manual is at Zen Garden.\\
      $v$ & The manual is at Earth Courtyard.\\
      $w$ & The manual is at Fire Courtyard.\\
    \end{tabularx}

    We have the following information.
    \begin{align}
      & p \lor q\\
      & r \lor s \implies p\\
      & \lnot r \implies q\\
      & t \implies \lnot p\\
      & p \implies u\\
      & q \implies v\\
      & r \implies w\\
      & t
    \end{align}
    We can infer the following.
    \[
      \begin{array}{c|l@{\hspace{50pt}}r}
        \text{Inference} & \text{Premises} & \\\hline
        \lnot p & (4) \land (8) & (9)\\
        q & (1) \land (9) & (10)\\
        v & (6) \land (10) & \\
      \end{array}
    \]
    Therefore the manual is at Earth Courtyard.
  \end{solution}
  
\end{questions}

\end{document}

%%% Local Variables:
%%% mode: latex
%%% TeX-master: t
%%% End:

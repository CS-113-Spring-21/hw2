\documentclass[addpoints]{exam}

\usepackage{amsmath}
\usepackage{amssymb}
\usepackage{geometry}
\usepackage{tabularx}

% Header and footer.
\pagestyle{headandfoot}
\runningheadrule
\runningfootrule
\runningheader{CS 113 Discrete Mathematics}{HW 2: Logic}{Spring 2020}
\runningfooter{}{Page \thepage\ of \numpages}{}
\firstpageheader{}{}{}

% \qformat{{\large\bf \thequestion. \thequestiontitle}\hfill[\totalpoints\ points]}
\boxedpoints
\printanswers

\title{Homework 2: Logic\\ CS 113 Discrete Mathematics\\ Habib University -- Spring 2020}
\author{Don't Grade Me}  % replace with your ID, e.g. oy02945
\date{}

\begin{document}
\maketitle

\begin{questions}
  
  
\question De Morgan's Laws from set theory have their counterparts in logic. Prove each of these using truth tables.
  \begin{parts}
  \part[5] Prove that $\neg(p \lor q) \equiv \neg p \land \neg q $.
    \begin{solution}
      % Complete  the solution.
      \[
        \begin{array}{c|*{6}{|c}}
          p & q & p \lor q & \neg(p \lor q) & \neg p & \neg q & \neg p \land \neg q \\
          \hline
          F & F & & & & \\
          F & T & & & & \\
          T & F & & & & \\
          T & T & & & &
        \end{array}
      \]
    \end{solution}
  \part[5] Prove that $\neg(p \land q) \equiv \neg p \lor \neg q $.
    \begin{solution}
      % Write your solution here
    \end{solution}
  \end{parts}

\question We want to write the statement, ``A person is popular only if they are cool or funny'', in propositional logic.
  \begin{parts}
  \part[5] Identify three simple propositions, $p, q, \text{ and } r$, needed for the representation and write out the corresponding expression that uses them to represent the given sentence.
    \begin{solution} % Complete the solution.
      The simple propositions are as follows.\\
      \begin{tabularx}{\textwidth}{l@{ : }X}
        $p$ & \\
        $q$ & \\
        $r$ &
      \end{tabularx}
      
      The expression is : .
    \end{solution}
  \part[5] For your expression identified above, write the converse, contrapositive, and inverse in propositional logic as well as complete English sentences.
    \begin{solution} % Complete the solution.
      The expression is : .\\
      \begin{tabularx}{\textwidth}{l|l|X}
        & Logical Notation & English sentence \\\hline\hline
        Converse & & \\\hline
        Contrapositive & & \\\hline
        Inverse & & 
      \end{tabularx}
    \end{solution}
  \end{parts}

\question[5] A small company makes widgets in a variety of constituent materials (aluminum, copper, iron), colors (red, green, blue, grey), and finishes (matte, textured, coated). Although there are more than one thousand possible combinations of widget features, the company markets only a subset of the possible combinations. The following sentences are constraints that characterize the possibilities. Suppose that a customer places an order for a copper widget that is both green and blue with a matte coating. Your job is to determine which constraints are satisfied and which are violated and provide an explanation.
  \begin{enumerate}
  \item aluminum $\lor$ copper $\lor$ iron
  \item aluminum $\implies$ grey
  \item copper $\land$ $\neg$ coated $\implies$ red
  \item coated $\land$ $\neg$ copper $\implies$ green
  \item green $\lor$ blue $\iff \neg$ textured $\land$ $\neg$ iron
  \end{enumerate}
  \begin{solution} % Complete the solution.
    In the table below, a value of 1 for a constraint indicates that the constraint is satisfied and a value of 0 indicates that it is violated.

    \begin{tabularx}{\textwidth}{l|l|X}
      Constraint & Satisfied & Explanation \\\hline
      aluminum $\lor$ copper $\lor$ iron & & \\
      aluminum $\implies$ grey & & \\
      copper $\land$ $\neg$ coated $\implies$ red & & \\
      coated $\land$ $\neg$ copper $\implies$ green & & \\
      green $\lor$ blue $\iff \neg$ textured $\land$ $\neg$ iron & & 
    \end{tabularx}
  \end{solution}
  
\question Consider the following statements.
  \begin{itemize}
  \item $ (p \lor q) \implies \neg r$
  \item $(\neg p \land \neg q) \land \neg r$
  \end{itemize}
  \begin{parts}
  \part[5] Prove their logical equivalence using a truth table.
    \begin{solution}
      % Write your solution here
    \end{solution}
  \part[5] If they are not equivalent, propose a statement obtained after minimally modification to the second one such that it is equivalent to the first. Prove the equivalence using identities.
    \begin{solution}
      % Write your solution here
    \end{solution}
  \end{parts}

  
\question A proposition that is always \textit{True} regardless of the truth values of the simpler propositions involved is called a \textit{tautology}. In other words, a tautology is a proposition that is logically equivalent to \textit{True}.
  \begin{parts}
  \part[5] Prove that $((p \implies q) \land (q \implies r)) \implies (p \implies r)$ is a tautology.
    \begin{solution}
      % Write your solution here
    \end{solution}

  \part[5] Given propositions, $p$ and $q$, $q$ is said to be \textit{inferred} from $p$ if $p \implies q$ is a tautology. Write a statement in English that correctly applies this new terminology to the expression in the previous part. Also provide an example in English by assigning suitable propositions to $p,q, \text{ and } r$.
    \begin{solution} % Complete the solution.
      A statement of the expression from the previous part is: . \\
      Using the following propositions for  $p,q, \text{ and } r$,\\
      \begin{tabularx}{\textwidth}{l@{ : }X}
        $p$ & \\
        $q$ & \\
        $r$ & \\
      \end{tabularx}
      the statement becomes: .
    \end{solution}
  \end{parts}


\question[5] You are given four cards each of which has a number on one side and a letter on another. You place them on a table in front of you and the four cards read: $A\ 5\ 2\ J$. Which cards would you turn over in order to test the following rule? 
  \begin{center}
    Cards with $5$ on one side have $J$ on the other side.
  \end{center}
  Explain your choice.
  \begin{solution} % Complete the solution.
    In the table below, a value of 1 for a card indicates that I will turn it and a value of 0 indicates that I will not turn it.

    \begin{tabularx}{\textwidth}{c|c|X}
      Card & Turned & Explanation \\\hline
      $A$ &  \\
      $5$ &  \\
      $2$ &  \\
      $J$ &  
    \end{tabularx}
  \end{solution}
  
\question An argument is said to be \textit{valid} if its \textit{conclusion} can be inferred from its \textit{premises}. An argument that is not valid is called an \textit{invalid} argument, or a \textit{fallacy}. For each of the arguments below, identify the simple propositions involved, write the premises and conclusion(s) in logical notation using the identified simple propositions, and decide whether it is valid. Justify your decision.
  \begin{parts}
  \part[5] If I am wealthy, then I am happy. I am happy, therefore, I am wealthy.
    \begin{solution}
      % Write your solution here
    \end{solution}
  \part[5]
    If Ahmed drives his car, he is at least 18 years old. Ahmed does not drive a car. Therefore, Ahmed is not yet 18 years old. 
    \begin{solution}
      % Write your solution here
    \end{solution}
  \part[5] If I study, then I will not fail CS 113. If I do not play cards too often, then I will study. I failed CS 113. Therefore, I played cards too often.
    \begin{solution}
      % Write your solution here
    \end{solution}
  \end{parts}

\question[5] One of your TA's has hidden a manual titled, ``Sacred Secrets: How to Earn an A+ and Keep your Mind'', somewhere on campus. As they could themselves not benefit from this manual, the directions they have left for you to find the manual are as follows.
  \begin{enumerate}
  \item There is a hint at Learn Courtyard or at the Gym.
  \item If your TA is sitting in Ehsas or they are absent, then there is a hint at Learn Courtyard.
  \item If your TA is not sitting in Ehsaas, then there is a hint at the Gym.
  \item If there are people in Learn Courtyard, then there is no hint at Learn Courtyard.
  \item If there is a hint at Learn Courtyard, then the manual is at Zen Garden.
  \item If there is hint at the Gym, then the manual is at Earth Courtyard.
  \item If your TA is absent, then the manual is at Fire Courtyard.
  \end{enumerate}
  You notice that there are people in Learn Courtyard. Where is the manual?

  Identify the relevant simple propositions to model the above in propositional logic. Represent the above situation using propositional logic and describe the steps needed to infer the location of the manual.
  \begin{solution}
    % Write your solution here
  \end{solution}
  
\end{questions}

\end{document}

%%% Local Variables:
%%% mode: latex
%%% TeX-master: t
%%% End:
